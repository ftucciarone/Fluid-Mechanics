\chapter{Review of set theory}
\begin{flushright}
\StartRef{book:FacchiniAlgebra}
\StartRef{book:KlenkeProbability}
\end{flushright}


Intuitively, a \emph{set}\index{Set} is a collection of items. This rather vague definition allows us to work with several objects while dealing with them as a whole. Sometimes, sets will be addressed with the alternative naming of \emph{family} or \emph{class}. The objects contained by a set are addressed as \emph{elements}\index{Element}, and their properties defines the characteristics of the set. Special sets, of common use in math, are
\begin{itemize}
\item $\N$, the set of natural numbers;
\item $\N^{*}$, the set of natural numbers different from zero;
\item $\Z$, the set of integers;
\item $\Q$, the set of rational numbers, as defined as $p/q$ where $p,q$ are integers and $q \neq 0$;
\item $\R$, the set of real numbers.
\end{itemize}

If $A$ is a set, and in the collection of elements that defines $A$ there is, among other elements, the element $x$, it is said that $x$ belongs to the set $A$, written as $x \in A$. Conversely, if the element $x$ is not in the collection defining $A$, it is said that $x$ does not belong to $A$, and is written $x \notin A$. For instance, if $x=0$, then $x \in \N$ but $x \notin \N^{*}$. The definition of $\Q$ could have been written also as $x \in \Q$ if $x=p/q$ with $p \in \Z$ and $q \in \N^{*}$.

If $A$ and $B$ are sets and the elements of $A$ are also contained, among others, in the set $B$, it is said that $A$ is a subset of $B$, written as $A \subseteq B$. It is clear that when $A \subseteq B$ and $x\in A$, then $x \in B$. An important remark is that every set is subset of itself, meaning that $A \subseteq A$, and it is called \emph{improper set}\index{Set!improper set} of $A$.

If $A$ and $B$ are sets and they are both defined by the same elements, the sets are called equal, and written as $A=B$. To show that two sets are equal, one has to prove that $A \subseteq B$ and $B \subseteq A$, that means that $x \in A$ if and only if $X \in B$. 

If $A$ is a subset of $B$ but there exist elements $y \in B$ such that $y \notin A$, then $A$ is called a  \emph{proper set} \index{Set!proper} of $B$, and it is denoted as $A \subset B$.

Finally, the \emph{empty set}\index{Set!empty} $\varnothing$ is the set that contains no element. Clearly, $x \notin \varnothing$ for all possible $x$ and $\varnothing \subseteq A$ for all possible sets $A$.

Sets may be denoted either by direct enumeration of their elements (when possible) between brackets, or by enunciation of the property that characterizes all the elements. The second way, the most useful, will be denoted with
\begin{equation*}
	A = \cbkt{x \colon . \, . \, .}
\end{equation*}
where $. . .$ will be substituted with the property defining the set. For instance, the rational set $\Q$ can be made explicit as
\begin{equation*}
	\Q = \cbkt{x \colon x=p/q, \, p,q \in \Z, \text{ and } q \neq 0}.
\end{equation*}

A final consideration is about the difference between $x$ and $\cbkt{x}$. The first one is an element, while $\cbkt{x}$ is the set containing $x$. In this perspective, sets may be visually identified with boxes, and elements with the things contained in the box. The box containing $x$ is $\cbkt{x}$, and it is of course different by the single element $x$. The empty set $\varnothing$ is an empty box. This visual identification of sets with boxes comes handy to define the \emph{power set}\index{Set!power set} of a set $A$, namely $\mathcal{P}\rbkt{A}$, that is the collection of all the sets that can be obtained with all the elements of $A$ in every possible rearrangement:
\begin{equation}
	\mathcal{P}\rbkt{A} = \cbkt{X \colon X \subseteq A}.
\end{equation}

\section{Operations between sets}
Let now $A$ and $B$ be sets. Between two sets  some basic operations can be defined and later generalised to more than two sets.
The \emph{union set}\index{Set!union} $A \cup B$ is the collection of elements $x$ that belongs to either $A$ \emph{or} $B$ (or both):
\begin{equation}
	A \cup B = \cbkt{x \colon x \in A \text{ or } x \in B}.
\end{equation}
The \emph{intersection set}\index{Set!intersection} $A \cap B$ is the collection of elements $x$ that belongs to $A$ \emph{and} $B$:
\begin{equation}
	A \cap B = \cbkt{x \colon x \in A \text{ and } x \in B}.
\end{equation}
Two sets for which $A \cap B = \varnothing$ are called \emph{disjoint}\index{Set!disjoint}.
The \emph{difference}\index{Set!difference} between $A$ and $B$, namely $A \setminus B$, is the set of all elements of $A$ that do not belong to $B$, that is:
\begin{equation}
	A \setminus B = \cbkt{x \colon x \in A \text{ and } x \notin B}.
\end{equation}
Lastly, the \emph{simmetric difference}\index{Set!simmetric difference} between $A$ and $B$, usually denoted with $A \bigtriangleup B$, is the set of all the elements $X$ that belongs either to both $A$ or $B$, but not both:
\begin{equation}
	A \bigtriangleup B = \cbkt{x \colon x \in A \text{ or } x \in B, x \notin A \cap B}.
\end{equation}
Two useful relations, that will not be proved in this document, are the so called De Morgan's laws\footnote{Adopting the notation of the overbar for the complementary set, that is $\setcmp{A}\equiv \overline{A}$, De Morgan's laws can be remembered as the rule \say{break the line, change the sign}, as indeed
\begin{equation*}
    \overline{\bigcap_{i=1}^{n}A_{i}}=\bigcup_{i=1}^{n}\overline{A_{i}} \qquad \text{and} \qquad
    \overline{\bigcup_{i=1}^{n}A_{i}}=\bigcap_{i=1}^{n}\overline{A_{i}}
\end{equation*}
}.
\begin{definition}[De Morgan's laws]
    \begin{equation}\label{eq:DeMorgan}
        \setcmp{\rbkt{\bigcap_{i=1}^{n}A_{i}}}&=\bigcup_{i=1}^{n}\rbkt{\setcmp{A_{i}}} \qquad \text{and} \qquad        \setcmp{\rbkt{\bigcup_{i=1}^{n}A_{i}}}&=\bigcap_{i=1}^{n}\rbkt{\setcmp{A_{i}}}
    \end{equation}
\end{definition}
De Morgan's laws describe the effect of complementation on unions and intersections, setting the basis to important properties of sets. In particular, De Morgan's laws establish the equivalence between closure with respect to unions and closure with respect to intersections. 

These trivial operations can be recast in a wider context. 
\begin{definition}[Operation]
    Let $A$ be a set. A $n-$ary \emph{operation}\index{Operation} on $A$ is an application
    \begin{equation*}
        \varpi\colon \underbrace{A\times ... \times A}_{n-\text{times}} \rightarrow A.
    \end{equation*}
    Operation can be then described as \textit{unary}, \textit{binary}, \textit{ternary} and so on, depending on the number of inputs.
\end{definition}
As an example, if $A$ is a set, then the intersection between elements of $A$ is an operation on $\mathscr{P}\rbkt{A}$ defined as 
\begin{align*}
    \cap\colon\mathscr{P}\rbkt{A}\times\mathscr{P}\rbkt{A}&\rightarrow\mathscr{P}\rbkt{A}\\
    \rbkt{X,Y}&\mapsto X\cap Y.
\end{align*}
The same holds for the union, difference and symmetric difference.
\begin{definition}[Equivalence relation]\index{Equivalence relation}
    A binary relation $\sim$ on a set $X$ is said to be an \emph{equivalence relation} if and only if it satisfies:
    \begin{enumerate}
        \item \emph{reflexivity}, $a \sim a$;
        \item \emph{symmetric}, $a \sim b$ and $b \sim a$;
        \item \emph{transitivity}, $a \sim b$, $b \sim c$, then $a \sim c$. 
    \end{enumerate}
    The equivalence class of $a$ under $\sim$, denoted by $\sbkt{a}$, is defined as
    \begin{equation*}
        \sbkt{a} = \cbkt{x\in X \colon x\sim a}.
    \end{equation*}
\end{definition}

\section{Algebraic structures \red{\footnotesize(just a list of useful things, for now...)}}
\begin{figure}
    \centering
    \begin{tikzpicture}[node distance=5cm,
                    every node/.style={fill=none, font=\sffamily}]
% Specification of nodes (position, etc.)
% This is standard to declare the nodes
%   \node (TAG)   [style, position]           {filling};
    % Boxes  
    \node (SigmaAlgebra) [AlgBase]                                            {{\textbf{\Large{$\bs\sigma-$algebra}} \footnotesize{
        \begin{itemize}
            \item $\Omega\in\mathscr{A}$;
            \item $A\in\mathscr{A} \Rightarrow \setcmp{A}\in\mathscr{A}$;
            \item $\rbkt{A_{n}}_{n\in\N}$ sequence of sets such that $A_{i}\in\mathscr{A} \, \forall i$, then \\ $\bigcup_{n=1}^{\infty}A_{n}\in\mathscr{A}$. 
        \end{itemize}}}};
    \node (Algebra)      [AlgBase, below of = SigmaAlgebra, xshift = -4.75cm]    {{\textbf{\Large{Algebra}} \footnotesize{
        \begin{itemize}
            \item $\varnothing,\Omega\in\mathcal{A}$;
            \item $A\in\mathcal{A} \Rightarrow \setcmp{A}\in\mathcal{A}$;
            \item $\rbkt{A_{1},A_{2},...,A_{N}}$ is a sequence of sets such that $A_{i}\in\mathcal{A} \, \forall i$ \\ then $\bigcup_{n=1}^{N}A_{n}\in\mathcal{A}$. 
        \end{itemize}}}};
    \node (SigmaRing)    [AlgBase, below of = SigmaAlgebra]                   {{\textbf{\Large{$\bs\sigma-$ring}} \footnotesize{
        \begin{itemize}
            \item $\varnothing\in\mathcal{A}$;
            \item $\forall A,B\in\mathcal{A}$ with $A\subset B$, then $A\setminus B\in\mathcal{A}$;
           \item $\rbkt{A_{n}}_{n\in\N}$ sequence of sets such that $A_{i}\in\mathcal{A} \, \forall i$, then  $\bigcup_{n=1}^{\infty}A_{n}\in\mathcal{A}$.
        \end{itemize}}}};
    \node (LambdaSystem) [AlgBase, below of = SigmaAlgebra, xshift =  4.75cm]    {{\textbf{\Large{$\bs\lambda-$system}} \footnotesize{
        \begin{itemize}
            \item $\Omega\in\mathcal{A}$;
            \item $\forall A,B\in\mathcal{A}$ with $A\subset B$, then $A\setminus B\in\mathcal{A}$;
            \item $ \forall A_{1},A_{2},...\in\mathcal{A}$ pairwise disjoint, $\bigcup_{n=1}^{\infty}A_{n}\in\mathcal{A}$.
        \end{itemize}}}};
    \node (Ring)         [AlgBase, below of = Algebra, xshift =  2cm]         {{\textbf{\Large{Ring}} \footnotesize{
        \begin{itemize}
            \item $\varnothing\in\mathcal{A}$;
            \item $\forall A,B\in\mathcal{A} \Rightarrow A\setminus B\in\mathcal{A}$;
            \item $\forall A,B\in\mathcal{A} \Rightarrow A\cup B\in\mathcal{A}$.
        \end{itemize}}}};
    \node (Semiring)     [AlgBase, below of = Ring]                           {{\textbf{\Large{Semiring}} \footnotesize{
        \begin{itemize}
            \item $\varnothing\in\mathcal{A}$;
            \item $\forall A,B\in\mathcal{A} \Rightarrow A\cap B\in\mathcal{A}$;
            \item $\forall A,B\in\mathcal{A}$, $A\setminus B$ is a finite union of mutually disjoint sets.
        \end{itemize}}}};

    % Formulas
    \node (SigmaCupStab1) [AlgForm,above of = Algebra, yshift = -2.5cm, xshift = 0cm]    {$\sigma-\cup-$stable};
    \node (OmegaInA1)     [AlgForm,above of = SigmaRing, yshift =  -2.5cm, xshift = 0.85cm]  {$\Omega\in\mathcal{A}$};
    \node (capStab)       [AlgForm,above of = LambdaSystem, yshift = -2.5cm, xshift = -0.75cm]{$\cap-$stable};
    \node (OmegaInA2)     [AlgForm,below of = Algebra, yshift =  2cm, xshift = 0.25cm]  {$\Omega\in\mathcal{A}$};
    \node (SigmaCupStab1) [AlgForm,below of = SigmaRing, yshift =  2cm, xshift = 0.0cm] {$\sigma-\cup-$stable};
    \node (cupStab)       [AlgForm,below of = Ring, yshift =  2cm, xshift = 1.1cm]      {$\cup-$stable};

  % Specification of Connections 
    \draw[->]([xshift=-0.0cm]Algebra.north)--([yshift=-0.0cm]SigmaAlgebra.south); 
    \draw[->]([xshift=-0.0cm]SigmaRing.north)--([yshift=-0.0cm]SigmaAlgebra.south); 
    \draw[->]([xshift=-0.0cm]LambdaSystem.north)--([yshift=-0.0cm]SigmaAlgebra.south); 
    
    \draw[->]([xshift=-0.0cm]Ring.north)--([yshift=-0.0cm]Algebra.south); 
    \draw[->]([xshift=-0.0cm]Ring.north)--([yshift=-0.0cm]SigmaRing.south); 
    
    \draw[->]([xshift=-0.0cm]Semiring.north)--([yshift=-0.0cm]Ring.south); 

\end{tikzpicture}
    \caption{Hierarchy of algebraic structures of interest. On top, the $\sigma-$algebra represents the structure that has the most fulfilling properties. A $\sigma-$algebra is of course an Algebra, a $\sigma-$ring and a $\lambda-$system, meaning that each structure represented here can be built from the underlying structures by adding some properties.}
    \label{fig:hierarchy}
\end{figure}

Algebraic structures are particular sets over which one (or more) operation is defined.
First, the notion of closure must be recalled, as the property of a set with respect to an operation. A set $X$ is called \textit{closed with respect to the operation} $\varpi$ if, for any $A,B\in X$ the result of the operation $A\varpi B$ is still contained in $X$. This property is also called stability.

\begin{definition}[$\pi-$system]
    Let $A$ be a set. The collection $\mathcal{A}\subseteq\mathscr{P}\rbkt{A}$ is called a $\bs{\pi-}$\emph{system}\index{$\pi$-system} if it is closed under finite intersections.
\end{definition}

\begin{definition}[Semiring]
    A collection of sets $\mathcal{A}\subset\mathscr{P}\rbkt{A}$ is called a \emph{semiring}\index{Semiring} if
    \begin{enumerate}[label=(\roman*)]
        \item $\varnothing\in\mathcal{A}$;
        \item for any $A,B\in\mathcal{A}$ then $A\cap B\in\mathcal{A}$;
        \item for any $A,B\in\mathcal{A}$ the difference between $A$ and $B$, namely $A\setminus B$, is a finite union of mutually disjoint sets.
    \end{enumerate}
\end{definition}
Taking sets of common use two example are easily built. The first sees $A=\R$, the collection of right-open intervals in $\R$, defined by $\mathcal{A}=\cbkt{\left[ a,b\right) \colon a\leq b, \,a,b\in\R }$, is a semiring, as well as its equivalent in $\R^{2}$, $\mathcal{A}=\cbkt{\left[ a,b\right)\times\left[ c,d\right) \colon a\leq b, c\leq d, \,a,b,c,d\in\R }$.

\begin{definition}[Ring]
    A collection of sets $\mathcal{A}\subset\mathscr{P}\rbkt{A}$ is called a \emph{ring}\index{Ring} if
    \begin{enumerate}[label=(\roman*)]
        \item $\varnothing\in\mathcal{A}$;
        \item for any $A,B\in\mathcal{A}$ then $A\setminus B\in\mathcal{A}$;
        \item for any $A,B\in\mathcal{A}$  then $A\cup B\in\mathcal{A}$.
    \end{enumerate}
\end{definition}

\begin{definition}[Algebra]\label{def:Algebra}
    Let $A$ be a set. The collection $\mathcal{A}\subseteq\mathscr{P}\rbkt{A}$ is called an \emph{algebra}\index{Algebra} (or \emph{field}\index{Field}) if:
    \begin{enumerate}[label=(\roman*)]
        \item $\varnothing,A\in\mathcal{A}$;
        \item if $A\in\mathcal{A}$ then $\setcmp{A}\in\mathcal{A}$;
        \item $\mathcal{A}$ is closed under finite intersections, that is
        if $\rbkt{A_{1},A_{2},...,A_{N}}$ is a sequence of sets such that $A_{i}\in\mathcal{A}$ for each $i$, then also $\bigcup_{n=1}^{N}A_{n}\in\mathcal{A}$. 
    \end{enumerate}
    In view of DeMorgan's laws $\eqref{eq:DeMorgan}$ the last condition can be restated in terms of finite intersections of members of $\mathcal{A}$. A shorter definition would be stating that an algebra is a non-empty set closed with respect to finite unions, finite intersections, and complements.
\end{definition}

\begin{definition}[$\lambda-$ system]\label{def:LambdaSystem}
    A collection of sets $\mathcal{A}\subset\mathscr{P}\rbkt{A}$ is called a \emph{$\bs\lambda-$ system}\index{$\lambda-$ system} if
    \begin{enumerate}[label=(\roman*)]
        \item $A\in\mathcal{A}$;
        \item for any $A,B\in\mathcal{A}$ with $A\subset B$, then $A\setminus B\in\mathcal{A}$;
        \item for any choice of pairwise disjoint sets $A_{1},A_{2},...\in\mathcal{A}$ the union $\bigcup_{n=1}^{\infty}A_{n}\in\mathcal{A}$.
    \end{enumerate}
\end{definition}

\begin{theorem}[Relations between algebraic structures]
    Between the algebraic structures introduced, the following relations can be proved:
    \begin{itemize}
        \item every $\sigma-$ring is a ring, every ring is a semiring;
        \item every algebra is a ring. An algebra on a finite set $\Omega$ is a $\sigma-$algebra.
    \end{itemize}
    These relations are summarized in figure \ref{fig:hierarchy}. The semiring is the structure with least properties, but can be improved with additional properties to build rings and moreover algebras or $\sigma-$rings. 
\end{theorem}

\section{$\bs\sigma-$algebras}
The previous structures are useful to lay the foundations for a definition of $\sigma-$algebra. This particular structure is at the very basis of measure theory and probability theory. This particular structure is indeed the simplest structure needed to define measure and information. This latter concept will be made clear in the following chapter \ref{chpt:ExpsAndProb}, while the former will be introduced at the end of this section (\blue{or hopefully in a future section about measure theory}).
\begin{definition}[$\sigma-$algebra]\label{def:sigmaAlgebra}\index{$\sigma-$algebra}
    We say that $\mathscr{A}\subseteq\mathscr{P}\rbkt{\Omega}$ is a  \emph{$\bs{\sigma-}$algebra} if 
    \begin{enumerate}[label=(\roman*)]
        \item\label{item:simaAlg1} $\Omega\in\mathscr{A}$;
        \item\label{item:simaAlg2} if $A\in\mathscr{A}$ then $\setcmp{A}\in\mathscr{A}$;
        \item\label{item:simaAlg3} if $\rbkt{A_{n}}_{n\in\N}$ is a sequence of sets such that $A_{i}\in\mathscr{A}$ for each $i$, then also $\bigcup_{n=1}^{\infty}A_{n}\in\mathscr{A}$. 
    \end{enumerate}
    Notice that \ref{item:simaAlg2} implies that $\varnothing\in\mathscr{A}$ since $\Omega\in\mathscr{A}$. Again, a shorter definition would be stating that an algebra is a non-empty set closed with respect to countable unions, countable intersections, and complements.
\end{definition}
\begin{theorem}[Relations between algebraic structures and $\sigma-$algebras.]
    Every $\sigma-$algebra is a $\lambda-$system, an algebra and a $\sigma-$ring.
    Figure \ref{fig:hierarchy} shows the $\sigma-$algebra as the structure on top, the one that has all the characteristics of the structures below.
\end{theorem}
We can start the discussion on $\sigma-$algebras by presenting four fundamental examples of a $\sigma-$algebra. Let $\Omega$ be a set, then:
\begin{enumerate}[label=(\roman*)]
    \item the set $\mathscr{A}_{T}=\cbkt{\varnothing,\Omega}$ is the \emph{trivial $\bs\sigma-$algebra}\index{$\sigma-$algebra!trivial}, meaning that it is composed by the simplest possible structures that one has;
    \item the power set $\mathscr{P}\rbkt{\Omega}$ is the \emph{gross $\bs\sigma-$algebra}\index{$\sigma-$algebra!gross}, since it contains all possible subsets of $\Omega$ is the largest $\sigma-$algebra that one can consider;
    \item given a set $A\in\Omega$, the $\sigma-$algebra generated by $A$ is defined as $\mathscr{A}=\cbkt{\varnothing,A,\setcmp{A},\Omega}$;
    \item given two sets $A,B\in\Omega$, a $\sigma-$algebra can be generated easily as
    \begin{equation*}
        \mathscr{A}=\cbkt{\varnothing,A,B,\setcmp{A},\setcmp{B},A\cup B,\Omega\setminus\rbkt{A\cup B},\Omega}.    
    \end{equation*}
    This provides a simple constructive way to produce $\sigma-$algebras, simply taking the generating sets, their complements, all their possible union and the complement of all the possible unions. 
\end{enumerate}
The last two examples shows how things become complicated when enlarging the number of generating sets. Taking more than three generating sets makes the explicit formulation of the $\sigma-$algebra unpractical. For this reason, another way to define $\sigma-$algebras in relation to their generators is necessary. To do such, the description of the algebraic properties of the $\sigma-$algebra are in order. In particular the following proposition.
\begin{proposition}
    The intersection of multiple $\sigma-$algebras is also a $\sigma-$algebra.
\end{proposition}
\begin{proof}
    Let $I$ be an arbitrary index set. Lets proceed by points to check that the intersection of $\sigma-$algebras $\mathscr{A}_{I}$, namely $\bigcap_{i\in I}\mathscr{A}_{i}$, is a $\sigma-$algebra:
    \begin{enumerate}[label=(\roman*)]
        \item each $\sigma-$algebra contains $\Omega$, their intersection will thus be non-empty and contains $\Omega$;
        \item if $A\in\bigcap_{i\in I}\mathscr{A}_{i}$ then it is a member of all the $\sigma-$algebras, therefore $\setcmp{A}$ is also a member of all the $\sigma-$algebras. It follows that $\setcmp{A}\in\bigcap_{i\in I}\mathscr{A}_{i}$.
        \item Let $\cbkt{A_{n}}_{n\in\N}$ be in the intersection of the $\sigma-$algebras $\bigcap_{i\in I}\mathscr{A}_{i}$. Then, for all $i\in I$ one has $\cbkt{A_{n}}_{n\in\N}\in\bigcap_{i\in I}\mathscr{A}_{i}$ and thus the union of all $A_{n}$ is still in each $\sigma-$algebra $\mathscr{A}_{i}$ for all the $i\in I$. Then
        \begin{equation*}
            \bigcap_{i\in I}\rbkt{\bigcup_{n\in\N}A_{n}}\in\bigcap_{i\in I}\mathscr{A}_{i} \quad \Rightarrow \quad \bigcup_{n\in\N}A_{n}\in\bigcap_{i\in I}\mathscr{A}_{i}
        \end{equation*}
        because the union $\bigcup_{n\in\N}A_{n}$ is one for all the $\mathscr{A}_{i}$.
    \end{enumerate}
\end{proof}
With this tool, creating new algebra is made in an implicit way by demanding that a certain set of sets, called \emph{generators}\index{$\sigma-$algebra!generators of} are included in the $\sigma-$algebra and the properties of such structure are respected. What will be done later on is to specify some properties on the generators and later taking into consideration the smallest $\sigma-$algebra that contains such sets, just like in usual linear algebra one specifies properties on the generators of the space and then construct every element of that space with this basis. It is thus needed the concept of \emph{smallest $\sigma-$algebra}\index{$\sigma-$algebra! smallest}.
\begin{definition}[Smallest $\sigma-$algebra]\label{def:SmallestSigmaAlgebra}
    Given a collection of subsets $\cbkt{A_{n}}\subseteq A$, where $n\in\Gamma$ and $\Gamma$ is not necessarily a countable index, the smallest $\sigma-$algebra $\mathscr{A}$ containing $A_{n}$ for all $n\in\Gamma$ is denoted by $\sigma\rbkt{\cbkt{A_{n}}}$, named the $\sigma-$algebra generated by the collection $\cbkt{A_{n}}$, that is
    \begin{equation}
        \sigma\rbkt{\cbkt{A_{n}}} = \bigcap\cbkt{\mathcal{G}\colon\mathcal{G}\subseteq\mathscr{P}\rbkt{A} \,\text{is a $\sigma-$algebra,}\, A_{n}\in\mathcal{G}\,\forall n \in\Gamma}.
    \end{equation}
\end{definition}
\begin{theorem}
    If two different sets of generators, $\cbkt{A_{n}}$ and $\cbkt{B_{m}}$, are such that $\cbkt{A_{n}}\in\sigma\rbkt{\cbkt{B_{m}}}$ and $\cbkt{B_{m}}\in\sigma\rbkt{\cbkt{A_{n}}}$, then $\sigma\rbkt{\cbkt{A_{n}}}=\sigma\rbkt{\cbkt{B_{m}}}$.
\end{theorem}
\blue{
What may not be clear now is why $\sigma-$algebras are so important, in which sense these structures are required to define a measure, and why measure is important in probability. This last point can be answered easily by stressing that probabilities are often evaluated as sums and integrals, with a particular weight (this will be clear in the following chapters), and thus a good measure theory is important to define sums and integral in a consistent and useful way. What about $\sigma-$algebras and measure then? Measure theory needs the concept of $\sigma-$algebras, as the three properties of definition \ref{def:sigmaAlgebra} are the most basics properties that allows you to measure (in some generalized sense) something. Suppose you want to measure something, that means taking a set and assigning it a value. Of course, the way you are assigning the value makes the most of the definition of measure, but you also need to be able of saying something about the sets. Lets focus a bit on the $\sigma-$algebra in an operational way.
Lets define the $\sigma-$algebra, $\mathscr{A}$, as a collection of sets that carries the possibility to measure (or relate) the sets in a common sense way. For now, this set $\mathscr{A}$ is empty. 
With no loss of generality in this reasoning exercise, lets suppose that you need to measure something finite, meaning that you are not going to measure the whole universe. This something will be your set $\Omega$, and is a collection of objects, and it makes sense to include it in $\mathscr{A}$ as it is the actual target of your measure. At some point, you want to say that something is empty, as much as that something is full, so makes sense to include also the empty set $\varnothing$ in $\mathscr{A}$. Some definitions of $\sigma-$algebra directly state that $\varnothing,\Omega\in\mathscr{A}$. Now, lets talk a bit about point \ref{item:simaAlg2} of the definition. Why should I need to include the complementary set of every set? The answer of this is actually one famous word: Eureka. Not always you will be able to measure something, so if you want your set $A$ to be in your list of measurable objects $\mathscr{A}$, maybe going directly to measure it is unfeasible and you need to approach it in some other way. So maybe, knowing that the complementary set is measurable and has some measure allows you to measure the original object with respect to the whole set and the complementary. Just like Archimedes did when measuring the weight of the crown of Gerone of Siracuse. Point three is conceptually easier, but stands on the same motivation. What if I'm not able to measure a set \say{as it is}? Well, then I decompose it in smaller pieces and the union of these pieces must lie in my set of measurable objects $\mathscr{A}$. Why having it infinite though? Well, not always finite sequences to the job. If I have to measure the area circle, but I don't exactly know how to evaluate $\pi$, I will need do construct a sequence of objects that give me some approximation of the area. A finite sequence will give me a good approximation if chosen correctly, but an infinite sequence is what I really need.} 
\section{Limits of sets}
In the previous section, it was discussed why there's the need of $\sigma-$algebras and why the definition involves an infinite sequence. In this section limits of sets will be discussed, as they will be used intensively in the following chapters.
\begin{definition}
    Let $\cbkt{A_{n}}$ be a sequence of sets, define
    \begin{align}
        \inf_{k\geq n} A_{k} &= \bigcap_{k=n}^{\infty}A_{k} \\
        \sup_{k\geq n} A_{K} &= \bigcup_{k=n}^{\infty}A_{k} \\
        \liminf_{n\rightarrow\infty} A_{n} &= \bigcup_{n\in\N}\bigcap_{k=n}^{\infty}A_{k} \\
        \limsup_{n\rightarrow\infty} A_{n} &= \bigcap_{n\in\N}\bigcup_{k=n}^{\infty}A_{k}.
    \end{align}
\end{definition}
Conceptually, the first two definition are easy to interpret\footnote{At least for the author.} while the last two are a bit more cumbersome. Lets dig into them a little more. A member of $\liminf_{n\rightarrow\infty} A_{n} =\bigcup_{n\in\N}\bigcap_{k=n}^{\infty}A_{k}$ is a member of at least one of the sets $\bigcap_{k=n}^{\infty}A_{k}$, meaning that it is a member \textit{either} of $A_{1}\cap A_{2}\cap A_{3}\cap \cdots$ \textit{or} $A_{2}\cap A_{3}\cap A_{4}\cap \cdots$ \textit{or} $A_{3}\cap A_{4}\cap A_{5}\cap \cdots$ and so on. That means that it is a member of all sets except for a finite amount of sets. Conversely, a member of $\limsup_{n\rightarrow\infty} A_{n}=\bigcap_{n\in\N}\bigcup_{k=n}^{\infty}A_{k}$ is a member of all the sets $\bigcup_{k=n}^{\infty}A_{k}$, so it is a member of $A_{1}\cup A_{2}\cup A_{3}\cup \cdots$ \textit{and} $A_{2}\cup A_{3}\cup A_{4}\cup \cdots$ \textit{and} $A_{3}\cup A_{4}\cup A_{5}\cup \cdots$ and so on. This means that it is a member of at least one of the sets that come later, i.e. a member of infinitely many sets $A_{j}$. This can expressed also with the following proposition.
\begin{proposition}
    Let $\cbkt{A_{n}}_{n\in\N}$ be a sequence of subsets of $\Omega$. Then
    \begin{align*}
        \limsup_{n\rightarrow\infty} A_{n} &= \cbkt{\omega\in\Omega \colon \sum_{n\in\N}I_{A_{n}} = \infty};\\
        \liminf_{n\rightarrow\infty} A_{n} &= \cbkt{\omega\in\Omega \colon \sum_{n\in\N}I_{A_{n}} < \infty}.
    \end{align*}
    From these two, it comes with no surprise that
    \begin{equation*}
        \liminf_{n\rightarrow\infty} \subset \limsup_{n\rightarrow\infty}.
    \end{equation*}
\end{proposition}


\begin{definition}[Limit of a sequence of sets]
    If a sequence of sets $\cbkt{A_{n}}$ admits $\liminf_{n\rightarrow\infty} A_{n}$, $\limsup_{n\rightarrow\infty} A_{n}$ and they coincide, then the \emph{limit}\index{Set!limit} of $\cbkt{A_{n}}$ is 
    \begin{equation*}
        \lim_{n\rightarrow\infty} A_{n} = \liminf_{n\rightarrow\infty} A_{n} = \limsup_{n\rightarrow\infty} A_{n}.
    \end{equation*}
\end{definition}

\begin{definition}[Monotonicity]
    A sequence of sets $\cbkt{A_{n}}_{n\in\N}$ is \emph{monotonically non-decreasing} if $A_{1}\subset A_{2}\subset A_{3}\subset\ldots$ and \emph{monotonically non-increasing} if $\ldots\subset A_{3}\subset A_{2}\subset A_{1}$. If a limit exist for such sequences and is such that $\lim_{n\rightarrow\infty}A_{n}=A$, these two kind of sequences are written as $A_{n}\nearrow A$ and $A_{n}\searrow A$.
\end{definition}

\begin{proposition}
    If $A_{n}\nearrow$ then $\lim_{n\rightarrow}A_{n}=\bigcup_{n\in\N}A_{n}$ and if $A_{n}\searrow$ then $\lim_{n\rightarrow}A_{n}=\bigcap_{n\in\N}A_{n}$.
\end{proposition}

