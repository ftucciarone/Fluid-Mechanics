\documentclass{article}
\usepackage[utf8]{inputenc}


\usepackage[italian,english]{babel}
\usepackage[utf8]{inputenc}
\usepackage{textcomp}

\setcounter{secnumdepth}{3}
\setcounter{tocdepth}{3}
\usepackage{emptypage}

\usepackage{titlepic}

\usepackage[toc,page]{appendix}

\usepackage{caption}
\usepackage{enumitem}     
\usepackage{lipsum}    

\usepackage{algorithm}
\usepackage{algpseudocode}
\usepackage{graphicx}

\usepackage{todonotes}

\usepackage{verbatim}
%%%%%%%%%%%%%%%%%%%%%%%%%%%%%%%%%%%%%%%%%%%%%%%%%
%       APPENDICE e subAppendici
%%%%%%%%%%%%%%%%%%%%%%%%%%%%%%%%%%%%%%%%%%%%%%%%%
\usepackage{appendix}
\usepackage{chngcntr}
\usepackage{etoolbox}
% Start of subappendices environment
\AtBeginEnvironment{subappendices}{%
\counterwithin{figure}{section}
\counterwithin{table}{section}
}

% End of subappendices environment
\AtEndEnvironment{subappendices}{%
\counterwithout{figure}{section}
\counterwithout{table}{section}
}

%%%%%%%%%%%%%%%%%%%%%%%%%%%%%%%%%%%%%%%%%%%%%%%%%
%       INDICE ANALITICO
%%%%%%%%%%%%%%%%%%%%%%%%%%%%%%%%%%%%%%%%%%%%%%%%%
\usepackage{makeidx}

%%%%%%%%%%%%%%%%%%%%%%%%%%%%%%%%%%%%%%%%%%%%%%%%%
%       Quotes like the frenchs like
%%%%%%%%%%%%%%%%%%%%%%%%%%%%%%%%%%%%%%%%%%%%%%%%%
\usepackage{dirtytalk}
 \usepackage[
 a4paper,                             % Format of the sheet
 twoside,                              % This set the format as a book
% inner=3cm,outer=2cm,top=2cm,bottom=3.5cm  
 ]{geometry}
%\usepackage{hyperref}

\usepackage{amsfonts}       % math fonts (tipo R per i reali)
\usepackage{amsmath}       % basic math package
\usepackage{amsthm}         % teoremi e corollari
\usepackage{mathtools}      % paired delimiters
\usepackage{mathrsfs}        % alternativa a mathcal
\usepackage{amssymb}       % \varsubsetneq
\usepackage{dsfont}           % \mathds{1}
\allowdisplaybreaks              %split align on different pages

\usepackage{cases}

\usepackage{xargs}

\usepackage{nicefrac}
%% Units of measurement (for math mode)
\usepackage{siunitx}


% To make subscript smaller upon request
\usepackage{scalerel} 
%%%%%%%%%%%%%%%%%%%%%%%%%%%%%%%%%%%%%%%%%%%%%%%%%%%%%%%%%%%%%%%
%       New abbreviations
%%%%%%%%%%%%%%%%%%%%%%%%%%%%%%%%%%%%%%%%%%%%%%%%%%%%%%%%%%%%%%%
\newcommand{\setcmp}[1]{{#1}^{\mathsf{c}}}
\newcommand{\given}{\,|\,}
\def\Rng{\text{Range}}
\newcommand\bs[1]{\boldsymbol{#1}}
\newcommand\sbcps[1]{\scaleto{\text{#1}}{4pt}}

%%%%%%%%%%%%%%%%%%%%%%%%%%%%%%%%%%%%%%%%%%%%%%%%%%%%%%%%%%%%%%%
%       New environments for theorems and much more
%%%%%%%%%%%%%%%%%%%%%%%%%%%%%%%%%%%%%%%%%%%%%%%%%%%%%%%%%%%%%%%
\newtheorem{theorem}{Theorem}[section]
\newtheorem{corollary}{Corollary}[theorem]
\newtheorem{lemma}[theorem]{Lemma}
\newtheorem*{remark}{Remark}
\theoremstyle{definition}
\newtheorem{definition}{Definition}[section]
\newtheorem{proposition}[theorem]{Proposition}
\newtheorem{examples}[theorem]{Examples}
\newtheorem{example}[theorem]{Example}
\newtheorem{relation}[theorem]{Relation}
\renewcommand\qedsymbol{$\blacksquare$}

%%%%%%%%%%%%%%%%%%%%%%%%%%%%%%%%%%%%%%%%%%%%%%%%%%%%%%%%%%%%%%%
%       Get the equation number
%%%%%%%%%%%%%%%%%%%%%%%%%%%%%%%%%%%%%%%%%%%%%%%%%%%%%%%%%%%%%%%
% \renewcommand{\theequation}{\thesubsection.\arabic{equation}}
  \renewcommand{\theequation}{\arabic{equation}}

%%%%%%%%%%%%%%%%%%%%%%%%%%%%%%%%%%%%%%%%%%%%%%%%%%%%%%%%%%%%%%%
%       New delimiters for easy parenthesis
%%%%%%%%%%%%%%%%%%%%%%%%%%%%%%%%%%%%%%%%%%%%%%%%%%%%%%%%%%%%%%%
\newcommand{\abs}[1]{\left\lvert #1 \right\rvert }    % \abs{b}  -->  |a,b|
\newcommand{\norm}[1]{\left\lVert #1 \right\rVert }   % \norm{a} --> ||a,b||
\newcommand{\rbkt}[1]{\left( #1 \right) }             % \rbkt{a,b} --> (a,b)
\newcommand{\sbkt}[1]{\left[ #1 \right] }             % \sbkt{a,b} --> [a,b]
\newcommand{\cbkt}[1]{\left\lbrace #1 \right\rbrace } % \cbkt{a,b} --> {a,b}
\newcommand{\bkt}[1]{\left\langle #1 \right\rangle }  % \bkt{a,b}  --> <a,b>

%%%%%%%%%%%%%%%%%%%%%%%%%%%%%%%%%%%%%%%%%%%%%%%%%%%%%%%%%%%%%%%
%       New commands for Math spaces
%%%%%%%%%%%%%%%%%%%%%%%%%%%%%%%%%%%%%%%%%%%%%%%%%%%%%%%%%%%%%%%
\def\N{\mathbb{ N}}
\def\Z{\mathbb{ Z}}
\def\Q{\mathbb{ Q}}
\def\R{\mathbb{ R}}
\def\C{\mathbb{ C}}
\def\D{\mathbb{ D}}
\def\S{\mathbb{ S}}
\def\I{\mathbb{ I}}
\def\P{\mathbb{ P}}
\def\E{\mathbb{ E}}
\def\O{\mathbb{ O}}
\def\T{\mathbb{ T}}
\def\one{\mathds{1}}

%%%%%%%%%%%%%%%%%%%%%%%%%%%%%%%%%%%%%%%%%%%%%%%%%%%%%%%%%%%%%%%
%       Defines the volume as V with a dash
%%%%%%%%%%%%%%%%%%%%%%%%%%%%%%%%%%%%%%%%%%%%%%%%%%%%%%%%%%%%%%%
\makeatletter
\DeclareRobustCommand{\Vol}{\text{\Volumedash}\hphantom{V}}
\newcommand{\Volumedash}{%
  \makebox[0pt][l]{%
    \ooalign{$V$\cr\raisebox{0.08em}{\kern0.08em--}\cr}
  }%
}
\makeatother

%%%%%%%%%%%%%%%%%%%%%%%%%%%%%%%%%%%%%%%%%%%%%%%%%%%%%%%%%%%%%%%
%       Defines the volume as v with a dash
%%%%%%%%%%%%%%%%%%%%%%%%%%%%%%%%%%%%%%%%%%%%%%%%%%%%%%%%%%%%%%%
\makeatletter
\DeclareRobustCommand{\vol}{\text{\volumedash}\hphantom{v}}
\newcommand{\volumedash}{%
  \makebox[0pt][l]{%
    \ooalign{$v$\cr\raisebox{-0.05em}{\kern0.04em--}\cr}
  }%
}
\makeatother

%%%%%%%%%%%%%%%%%%%%%%%%%%%%%%%%%%%%%%%%%%%%%%%%%%%%%%%%%%%%%%%
%       Various types of convergences
%%%%%%%%%%%%%%%%%%%%%%%%%%%%%%%%%%%%%%%%%%%%%%%%%%%%%%%%%%%%%%%
\newcommand{\lpconv}[1][p]{%  L^p convergence
    \xrightarrow{\makebox[2em][c]{$\scriptstyle L^{#1}$}}
    } 
\newcommand{\prconv}[1]{%  Probability convergence
    \xrightarrow{\makebox[2em][c]{$\scriptstyle P$}}
    }
\newcommand{\asconv}[1][p]{%  Almost sure convergence
    \xrightarrow{\makebox[2em][c]{$\scriptstyle a.s.$}}
    }

%%%%%%%%%%%%%%%%%%%%%%%%%%%%%%%%%%%%%%%%%%%%%%%%%%%%%%%%%%%%%%%
%       Stackrel for long stacks
%%%%%%%%%%%%%%%%%%%%%%%%%%%%%%%%%%%%%%%%%%%%%%%%%%%%%%%%%%%%%%%      
\newlength{\leftstackrelawd}
\newlength{\leftstackrelbwd}
\def\leftstackrel#1#2{\settowidth{\leftstackrelawd}%
{${{}^{#1}}$}\settowidth{\leftstackrelbwd}{$#2$}%
\addtolength{\leftstackrelawd}{-\leftstackrelbwd}%
\leavevmode\ifthenelse{\lengthtest{\leftstackrelawd>0pt}}%
{\kern-.5\leftstackrelawd}{}\mathrel{\mathop{#2}\limits^{#1}}}      
      
%%%%%%%%%%%%%%%%%%%%%%%%%%%%%%%%%%%%%%%%%%%%%%%%%%%%%%%%%%%%%%%
%       Average integrals with horizontal mean bar
%%%%%%%%%%%%%%%%%%%%%%%%%%%%%%%%%%%%%%%%%%%%%%%%%%%%%%%%%%%%%%%      
\def\Xint#1{\mathchoice
{\XXint\displaystyle\textstyle{#1}}%
{\XXint\textstyle\scriptstyle{#1}}%
{\XXint\scriptstyle\scriptscriptstyle{#1}}%
{\XXint\scriptscriptstyle%
\scriptscriptstyle{#1}}%
\!\int}
\def\XXint#1#2#3{{\setbox0=\hbox{$#1{#2#3}{%
\int}$ }
\vcenter{\hbox{$#2#3$ }}\kern-.6\wd0}}
\def\ddashint{\Xint=}
\def\dashint{\Xint-}

%%%%%%%%%%%%%%%%%%%%%%%%%%%%%%%%%%%%%%%%%%%%%%%%%%%%%%%%%%%%%%%
%       Life is better with some colors
%%%%%%%%%%%%%%%%%%%%%%%%%%%%%%%%%%%%%%%%%%%%%%%%%%%%%%%%%%%%%%%
\usepackage{xcolor}
\newcommand*{\red}{\textcolor{red}}
\newcommand*{\blue}{\textcolor{blue}}
\newcommand*{\yellow}{\textcolor{yellow}}
\newcommand*{\green}{\textcolor{green}}

%%%%%%%%%%%%%%%%%%%%%%%%%%%%%%%%%%%%%%%%%%%%%%%%%%%%%%%%%%%%%%%
%       Missing References to highlighted in boldface red
%%%%%%%%%%%%%%%%%%%%%%%%%%%%%%%%%%%%%%%%%%%%%%%%%%%%%%%%%%%%%%%
\newcommand*{\missingreference}{{\color{red}\bfseries ??missing??}}
\makeatletter
\def\@setref#1#2#3{%
   \ifx#1\relax
    \protect\G@refundefinedtrue
    \nfss@text{\reset@font\missingreference}%
    \@latex@warning{Reference `#3' on page \thepage \space
              undefined}%
   \else
    \expandafter#2#1\null
   \fi}
\makeatother
%%%%%%%%%%%%%%%%%%%%%%%%%%%%%%%%%%%%%%%%%%%%%%%%%%%%%%%%%%%%%%%
%       Missing Citation to highlighted in boldface red
%%%%%%%%%%%%%%%%%%%%%%%%%%%%%%%%%%%%%%%%%%%%%%%%%%%%%%%%%%%%%%%
\newcommand*{\missingcitation}{{\color{red}\bfseries ?missing?}}
\makeatletter
\def\@citex[#1]#2{\leavevmode
   \let\@citea\@empty
   \@cite{\@for\@citeb:=#2\do
     {\@citea\def\@citea{,\penalty\@m\ }%
      \edef\@citeb{\expandafter\@firstofone\@citeb\@empty}%
      \if@filesw\immediate\write\@auxout{\string\citation{\@citeb}}\fi
      \@ifundefined{b@\@citeb}{\hbox{\reset@font\missingcitation}%
        \G@refundefinedtrue
        \@latex@warning
          {Citation `\@citeb' on page \thepage \space undefined}}%
        {\@cite@ofmt{\csname b@\@citeb\endcsname}}}}{#1}}
\makeatother


%%%%%%%%%%%%%%%%%%%%%%%%%%%%%%%%%%%%%%%%%%%%%%%%%%%%%%%%%%%%%%%%%%%%%%%%%%%
%   PgfPlots
%%%%%%%%%%%%%%%%%%%%%%%%%%%%%%%%%%%%%%%%%%%%%%%%%%%%%%%%%%%%%%%%%%%%%%%%%%%
\usepackage{pgfplots}
\pgfplotsset{compat=newest}
\usepgfplotslibrary{fillbetween,
                }




%%%%%%%%%%%%%%%%%%%%%%%%%%%%%%%%%%%%%%%%%%%%%%%%%%%%%%%%%%%%%%%%%%%%%%%%%%%
%   Tikz
%%%%%%%%%%%%%%%%%%%%%%%%%%%%%%%%%%%%%%%%%%%%%%%%%%%%%%%%%%%%%%%%%%%%%%%%%%%
\usepackage{tikz}
\usetikzlibrary{calc,
                trees,
                positioning,
                arrows,
                chains,
                shapes.geometric,
                decorations.pathreplacing,
                decorations.pathmorphing,
                shapes,
                matrix,shapes.symbols}
    
    
 \usetikzlibrary{arrows.meta}
\usepackage{forest}   
    
%% Classification of tiks marks
\def\showmarkEXPFour{\tikz[baseline=-0.55ex]\node[black,mark size=0.5ex]{\pgfuseplotmark{square*}};\hspace{0.75pt}}
\def\showmarkEXPFive{\tikz[baseline=-0.55ex]\node[black,mark size=0.5ex]{\pgfuseplotmark{square}};\hspace{0.75pt}}
\def\showmarkEXPSeven{\tikz[baseline=-0.55ex]\node[blue,mark size=0.5ex]{\pgfuseplotmark{square}};\hspace{0.75pt}}
\def\showmarkEXPTen{\tikz[baseline=-0.55ex]\node[blue,mark size=0.5ex]{\pgfuseplotmark{halfdiamond*}};\hspace{0.75pt}}


%%%%%%%%%%%%%%%%%%%%%%%%%%%%%%%%%%%%%%%%%%%%%%%%%%%%%%%%%%%%%%%%%%%%%%%%%%%
%   Tikz
%%%%%%%%%%%%%%%%%%%%%%%%%%%%%%%%%%%%%%%%%%%%%%%%%%%%%%%%%%%%%%%%%%%%%%%%%%%
\usepackage{xcolor}
\usepackage{textcomp}

\usetikzlibrary{positioning,fit,shapes,calc,arrows,arrows.meta,fadings,through}



\definecolor{excel}{RGB}{50,147,104}
\definecolor{txt}{RGB}{0,104,232}

\definecolor{left}{RGB}{237,237,235}
\definecolor{right}{RGB}{245,245,244}
\definecolor{win}{RGB}{245,245,244}

\tikzset{%
        >={Latex[width=2mm,length=2mm]},
             % Specifications for style of nodes:
            base/.style = {rectangle, rounded corners, draw=white,
                                                minimum width=4cm, minimum height=1cm,
                                                text centered, font=\sffamily},
            input_txt/.style = {rectangle, rounded corners, draw=white,
                                                minimum width=2cm, minimum height=1cm,
                                                text centered, font=\sffamily},
            input_excel/.style = {rectangle, rounded corners, draw=white,
                                                minimum width=2cm, minimum height=1cm,
                                                text centered, font=\sffamily,},
            activityStarts/.style = {base, fill=blue!30},
            excel/.style = {input_excel, fill=excel!30},
            txt/.style = {input_txt, fill=txt!30},
            xml/.style = {input_txt, fill=txt!30},
            startstop/.style = {base, fill=red!30},
            activityRuns/.style = {base, fill=green!30},
            process/.style = {base, minimum width=2.5cm, fill=orange!15},
            sampling/.style = {base, minimum width=2.5cm, fill=green!15},
            null/.style = {base, minimum width=2.5cm, draw=black},
            block/.style = {rectangle, draw, fill=blue!20, 
                                               text width=15em, text centered, rounded corners, minimum height=4em},
            struct/.style = {rectangle, draw,
                                               text width=5.5em, align=left, rounded corners},
            substruct/.style = {rectangle,% draw,
                                               text width=6em,align=left, rounded corners},
            mfunc/.style = {rectangle, %draw,
                                                fill=blue!20, text centered, rounded corners, minimum height=4em},
            specs/.style = {rectangle, %draw, 
                            align=left, rounded corners},
            line/.style = {draw, -latex'}
           }




\forestset{%
  declare toks={my label}{},
  probability tree/.style={%
    for tree={%
      math content,
      grow'=0,
      child anchor=parent,
      l sep+=40pt,
      +edge={every node/.append style={sloped, midway}},
      delay={%
        split option={my label}{:}{above edge, below edge},
      }
    },
    where n children=0{%
      +edge={-{Circle[width=3pt,length=3pt]}},
      if n=1{
        !u.s sep+=30pt,
      }{}
    }{%
      text width=50pt,
      text centered,
      if={isodd(n_children)}{%
        for n/.wrap pgfmath arg={{##1}{calign with current edge}}{int(.5*(n_children()+1))}
      }{},
    },
    above edge/.style={%
      +edge label={node [above] {$##1$}},
    },
    below edge/.style={%
      +edge label={node [below] {$##1$}},
    },
  },
}
%%%%%%%%%%%%%%%%%%%%%%%%%%%%%%%%%%%%%%%%%%%%%%%%%
%       Bibliography
%%%%%%%%%%%%%%%%%%%%%%%%%%%%%%%%%%%%%%%%%%%%%%%%%
\usepackage{babel,csquotes,xpatch}%   recommended 
\usepackage[backend=biber]{biblatex} 

%%%%%%%%%%%%%%%%%%%%%%%%%%%%%%%%%%%%%%%%%%%%%%%%%
%       Bibliography entries
%%%%%%%%%%%%%%%%%%%%%%%%%%%%%%%%%%%%%%%%%%%%%%%%%
\bibliography{./Biblio/Biblio.bib}

%%%%%%%%%%%%%%%%%%%%%%%%%%%%%%%%%%%%%%%%%%%%%%%%%
%       Custom Quotes (so you can rebuild them if
%         you change package)
%%%%%%%%%%%%%%%%%%%%%%%%%%%%%%%%%%%%%%%%%%%%%%%%%
\newcommand\inlinecite[1]{%           Author et al. (year) 
  \citeauthor{#1}~(\citeyear{#1})}
  
\newcommand\StartRef[1]{%             Author et al. (year),[entry number]
    \citeauthor{#1}~(\citeyear{#1}),\cite{#1}\\}  


\title{Packages\_Templates}
\author{Francesco Tucciarone}

\begin{document}

\maketitle

This document purpose is just to set the packages that I will always use in such a way they can be linked in all the other projects.


\section{Math packages and utilities}
Complete list of math packages and options of common use:

\texttt{amsfonts}: Standard math fonts, like $\mathbb{R}$ for real numbers.

\texttt{amsmath}: Standard math package.

\texttt{amsthm}: Theorems and corollaries.

\texttt{mathtools}: Paired delimiters.

\texttt{mathrsfs}: Fancy calligraphic math symbols $\mathscr{A}$.

\texttt{amssymb}: Extended math symbols, like $\varsubsetneq$.

\texttt{dsfont}: Fancy math fonts, like $\mathds{R}$.

\texttt{\textbackslash allowdisplaybreaks}: Allow to split align on different pages

\texttt{cases}: For fancy equation with cases.

\texttt{xargs}: Extended version of \texttt{\textbackslash newcommand}.

\texttt{nicefrac}: Package for fractions in the form of $\nicefrac{a}{b}$.

%% Units of measurement (for math mode)
\texttt{siunitx}: Nice package for units of measure.

% To make subscript smaller upon request
\texttt{scalerel}: To redefine the dimension of the subscripts upon request.



\section{Newcommands}
\subsection{Abbreviation and other things}
\begin{minipage}[t]{0.78\textwidth}
    \begin{verbatim}
    %%%%%%%%%%%%%%%%%%%%%%%%%%%%%%%%%%%%%%%%%%%%%%%%%%%%%%%%%%%%%%%
    %       New abbreviations
    %%%%%%%%%%%%%%%%%%%%%%%%%%%%%%%%%%%%%%%%%%%%%%%%%%%%%%%%%%%%%%%
    \newcommand{\setcmp}[1]{{#1}^{\mathsf{c}}}
    \newcommand{\given}{\,|\,}
    \def\Rng{\text{Range}}
    \newcommand\bs[1]{\boldsymbol{#1}}
    \newcommand\sbcps[1]{\scaleto{\text{#1}}{4pt}}
    \end{verbatim}
\end{minipage}
\begin{minipage}[t]{0.28\textwidth}
    \null\null\null
    \verb!\setcmp{A}!$=\setcmp{A}$\newline
    \verb!A\given B !$=A\given B$\newline
    \verb!\Rng      !$=\Rng$\newline
    \verb!\bs{\xi}  !$=\boldsymbol{\xi}$\newline 
    Fix height for capital subscript
\end{minipage}
{\color{red}\textbf{NOTE:} if $\bs{\xi}$ and $\boldsymbol{\xi}$ are not equal  then there is a conflict with \verb!\usepackage{layout}!. The package \texttt{layout} however is needed just to print the layout settings to define the page geometry and nothing more.}
\subsection{Theorems and other environments}
\begin{verbatim}
    %%%%%%%%%%%%%%%%%%%%%%%%%%%%%%%%%%%%%%%%%%%%%%%%%%%%%%%%%%%%%%%
    %       New environments for theorems and much more
    %%%%%%%%%%%%%%%%%%%%%%%%%%%%%%%%%%%%%%%%%%%%%%%%%%%%%%%%%%%%%%%
    \newtheorem{theorem}{Theorem}[section]
    \newtheorem{corollary}{Corollary}[theorem]
    \newtheorem{lemma}[theorem]{Lemma}
    \newtheorem*{remark}{Remark}
    \theoremstyle{definition}
    \newtheorem{definition}{Definition}[section]
    \newtheorem{proposition}[theorem]{Proposition}
    \newtheorem{examples}[theorem]{Examples}
    \newtheorem{example}[theorem]{Example}
    \newtheorem{relation}[theorem]{Relation}
    \renewcommand\qedsymbol{$\blacksquare$} % black square Q.E.D.
\end{verbatim}

\subsection{Getting the number of the current equation}
\begin{verbatim}
    %%%%%%%%%%%%%%%%%%%%%%%%%%%%%%%%%%%%%%%%%%%%%%%%%%%%%%%%%%%%%%%
    %       Get the equation number
    %%%%%%%%%%%%%%%%%%%%%%%%%%%%%%%%%%%%%%%%%%%%%%%%%%%%%%%%%%%%%%%
    % \renewcommand{\theequation}{\thesubsection.\arabic{equation}}
      \renewcommand{\theequation}{\arabic{equation}}
\end{verbatim}

\subsection{Delimiters for easy parenthesis}
\begin{minipage}[t]{0.78\textwidth}
    \begin{verbatim}
    %%%%%%%%%%%%%%%%%%%%%%%%%%%%%%%%%%%%%%%%%%%%%%%%%%%%%%%%%%%%%%%
    %       New delimiters for easy parenthesis
    %%%%%%%%%%%%%%%%%%%%%%%%%%%%%%%%%%%%%%%%%%%%%%%%%%%%%%%%%%%%%%%
    \newcommand{\abs }[1]{\left\lvert #1 \right\rvert }
    \newcommand{\norm}[1]{\left\lVert #1 \right\rVert }
    \newcommand{\rbkt}[1]{\left( #1 \right) }
    \newcommand{\sbkt}[1]{\left[ #1 \right] }
    \newcommand{\cbkt}[1]{\left\lbrace #1 \right\rbrace }
    \newcommand{\bkt }[1]{\left\langle #1 \right\rangle }
    \end{verbatim}
\end{minipage}
\begin{minipage}[t]{0.18\textwidth}
    \null\null\null
    \verb!\abs{a} !$=\abs{a}$\newline
    \verb!\norm{a}!$=\norm{a}$\newline
    \verb!\rbkt{a}!$=\rbkt{a}$\newline
    \verb!\sbkt{a}!$=\sbkt{a}$\newline
    \verb!\cbkt{a}!$=\cbkt{a}$\newline
    \verb!\bkt{a} !$=\bkt{a}$\newline
\end{minipage}

\subsection{Mathematical notation for spaces}
\begin{minipage}[t]{0.78\textwidth}
    \begin{verbatim}
    %%%%%%%%%%%%%%%%%%%%%%%%%%%%%%%%%%%%%%%%%%%%%%%%%%%%%%%%%%%%%%%%
    %       New commands for Math spaces
    %%%%%%%%%%%%%%%%%%%%%%%%%%%%%%%%%%%%%%%%%%%%%%%%%%%%%%%%%%%%%%%%
    \def\N{\mathbb{ N}}
    \def\Z{\mathbb{ Z}}
    \def\Q{\mathbb{ Q}}
    \def\R{\mathbb{ R}}
    \def\C{\mathbb{ C}}
    
    \def\D{\mathbb{ D}}
    \def\S{\mathbb{ S}}
    \def\I{\mathbb{ I}}
    \def\P{\mathbb{ P}}
    \def\E{\mathbb{ E}}
    \def\O{\mathbb{ O}}
    \def\T{\mathbb{ T}}
    \def\one{\mathds{1}}
    \end{verbatim}
\end{minipage}
\begin{minipage}[t]{0.18\textwidth}
    \null\null\null
    \verb!\N !$=\N$\newline
    \verb!\Z !$=\Z$\newline
    \verb!\Q !$=\Q$\newline
    \verb!\R !$=\R$\newline
    \verb!\C !$=\C$\newline
    \null\newline
    \verb!\D !$=\D$\newline
    \verb!\S !$=\S$\newline
    \verb!\I !$=\I$\newline
    \verb!\P !$=\P$\newline
    \verb!\E !$=\E$\newline
    \verb!\O !$=\O$\newline
    \verb!\T !$=\T$\newline
    \verb!\one !$=\one$\newline
\end{minipage}


\subsection{Volumes with the dash}
\begin{minipage}[t]{0.78\textwidth}
    \begin{verbatim}
    % Volume with the little dash over
    %%%%%%%%%%%%%%%%%%%%%%%%%%%%%%%%%%%%%%%%%%%%%%%%%%%%%%%%%%%%%%%%
    %       Defines the volume as V with a dash
    %%%%%%%%%%%%%%%%%%%%%%%%%%%%%%%%%%%%%%%%%%%%%%%%%%%%%%%%%%%%%%%%
    \makeatletter
        \DeclareRobustCommand{\Vol}{\text{\Volumedash}\hphantom{V}}
        \newcommand{\Volumedash}{%
        \makebox[0pt][l]{%
            \ooalign{$V$\cr\raisebox{0.08em}{\kern0.08em--}\cr}
        }%
    }
    \makeatother
    \end{verbatim}
\end{minipage}
\begin{minipage}[t]{0.18\textwidth}
    \begin{equation*}
        \scalebox{10}{\Vol}
    \end{equation*}
\end{minipage}\newline
\begin{minipage}[t]{0.78\textwidth}
    \begin{verbatim}
    %%%%%%%%%%%%%%%%%%%%%%%%%%%%%%%%%%%%%%%%%%%%%%%%%%%%%%%%%%%%%%%%
    %       Defines the volume as v with a dash
    %%%%%%%%%%%%%%%%%%%%%%%%%%%%%%%%%%%%%%%%%%%%%%%%%%%%%%%%%%%%%%%%
    \makeatletter
        \DeclareRobustCommand{\vol}{\text{\volumedash}\hphantom{v}}
        \newcommand{\volumedash}{%
        \makebox[0pt][l]{%
            \ooalign{$v$\cr\raisebox{-0.05em}{\kern0.04em--}\cr}
        }%
    }
    \makeatother
    \end{verbatim}
\end{minipage}
\begin{minipage}[t]{0.18\textwidth}
    \begin{equation*}
        \scalebox{10}{\vol}
    \end{equation*}
\end{minipage}

\section{Convergences}
\begin{minipage}[t]{0.78\textwidth}
    \begin{verbatim}
    %%%%%%%%%%%%%%%%%%%%%%%%%%%%%%%%%%%%%%%%%%%%%%%%%%%%%%%%%%%%%%%
    %       Various types of convergences
    %%%%%%%%%%%%%%%%%%%%%%%%%%%%%%%%%%%%%%%%%%%%%%%%%%%%%%%%%%%%%%%
    \newcommand{\lpconv}[1][p]{          %          L^p convergence
        \xrightarrow{\makebox[2em][c]{$\scriptstyle L^{#1}$}}} 
    \newcommand{\prconv}[1]{             %  Probability convergence
        \xrightarrow{\makebox[2em][c]{$\scriptstyle P$}}}
    \newcommand{\asconv}[1][p]{          %  Almost sure convergence
        \xrightarrow{\makebox[2em][c]{$\scriptstyle a.s.$}}}
    \end{verbatim}
\end{minipage}
\begin{minipage}[t]{0.18\textwidth}
    \null\null\null
    \verb!\lpconv ! $ =\lpconv $ \newline
    \null\newline
    \verb!\prconv ! $ =\prconv $ \newline
    \null\newline
    \verb!\asconv ! $ =\asconv $ \newline
\end{minipage}\newline

\subsection{Integrals with symbols over}
\begin{minipage}[t]{0.78\textwidth}
    \begin{verbatim}
    %%%%%%%%%%%%%%%%%%%%%%%%%%%%%%%%%%%%%%%%%%%%%%%%%%%%%%%%%%%%%%%
    %       Average integrals with horizontal mean bar
    %%%%%%%%%%%%%%%%%%%%%%%%%%%%%%%%%%%%%%%%%%%%%%%%%%%%%%%%%%%%%%%      
    \def\Xint#1{\mathchoice
    {\XXint\displaystyle\textstyle{#1}}%
    {\XXint\textstyle\scriptstyle{#1}}%
    {\XXint\scriptstyle\scriptscriptstyle{#1}}%
    {\XXint\scriptscriptstyle%
    \scriptscriptstyle{#1}}%
    \!\int}
    \def\XXint#1#2#3{{\setbox0=\hbox{$#1{#2#3}{%
    \int}$ }
    \vcenter{\hbox{$#2#3$ }}\kern-.6\wd0}}
        \def\ddashint{\Xint=}
        \def\dashint{\Xint-}
    \end{verbatim}
\end{minipage}
\begin{minipage}[t]{0.18\textwidth}
    \null\null\null
    \verb!\Xint{arg} ! $=\Xint{arg}$ \\
    \verb!\ddashint  ! $=\ddashint$\newline
    \verb!\dashint   ! $=\dashint$\newline
\end{minipage}\newline

\subsection{Stackrel modified for longer stacks}
\begin{minipage}[t]{0.78\textwidth}
    \begin{verbatim}
    %%%%%%%%%%%%%%%%%%%%%%%%%%%%%%%%%%%%%%%%%%%%%%%%%%%%%%%%%%%%%%%
    %       Stackrel for long stacks
    %%%%%%%%%%%%%%%%%%%%%%%%%%%%%%%%%%%%%%%%%%%%%%%%%%%%%%%%%%%%%%%      
    \newlength{\leftstackrelawd}
    \newlength{\leftstackrelbwd}
    \def\leftstackrel#1#2{\settowidth{\leftstackrelawd}%
    {${{}^{#1}}$}\settowidth{\leftstackrelbwd}{$#2$}%
    \addtolength{\leftstackrelawd}{-\leftstackrelbwd}%
    \leavevmode\ifthenelse{\lengthtest{\leftstackrelawd>0pt}}%
    {\kern-.5\leftstackrelawd}{}\mathrel{\mathop{#2}\limits^{#1}}}  
    \end{verbatim}
\end{minipage}
\begin{minipage}[t]{0.18\textwidth}
    This is used to avoid
    \begin{align*}
        A &= \text{equation...} \\
        &\stackrel{looong}{=}\text{equation...}
    \end{align*}
    and produce instead a centered equal sign
    \begin{align*}
        A &= \text{equation...} \\
        &\leftstackrel{looong}{=}\text{equation...}
    \end{align*}
\end{minipage}\newline







\section{Highlighted missing reference}
\begin{minipage}[t]{0.78\textwidth}
    \begin{verbatim}
    %%%%%%%%%%%%%%%%%%%%%%%%%%%%%%%%%%%%%%%%%%%%%%%%%%%%%%%%%%%%%%%%
    %       Missing References to highlighted in boldface red
    %%%%%%%%%%%%%%%%%%%%%%%%%%%%%%%%%%%%%%%%%%%%%%%%%%%%%%%%%%%%%%%%
    \newcommand*{\missingreference}{
                {\color{red}\bfseries ??missing??}}
    \makeatletter
    \def\@setref#1#2#3{%
       \ifx#1\relax
        \protect\G@refundefinedtrue
        \nfss@text{\reset@font\missingreference}%
        \@latex@warning{Reference `#3' on page \thepage \space
                  undefined}%
       \else
        \expandafter#2#1\null
       \fi}
    \makeatother
    \end{verbatim}
\end{minipage}
\begin{minipage}[t]{0.18\textwidth}
    This is a missing reference: \ref{somefig}.
\end{minipage}
\newline
\begin{minipage}[t]{0.78\textwidth}
    \begin{verbatim}
    %%%%%%%%%%%%%%%%%%%%%%%%%%%%%%%%%%%%%%%%%%%%%%%%%%%%%%%%%%%%%%%%
    %       Missing Citation to highlighted in boldface red
    %%%%%%%%%%%%%%%%%%%%%%%%%%%%%%%%%%%%%%%%%%%%%%%%%%%%%%%%%%%%%%%%
    \newcommand*{\missingcitation}{
                {\color{red}\bfseries ?missing?}}
    \makeatletter
    \def\@citex[#1]#2{\leavevmode
        \let\@citea\@empty
        \@cite{\@for\@citeb:=#2\do
            {\@citea\def\@citea{,\penalty\@m\ }%
            \edef\@citeb{\expandafter\@firstofone\@citeb\@empty}%
            \if@filesw\immediate\write\@auxout{
                            \string\citation{\@citeb}}\fi
            \@ifundefined{b@\@citeb}{
                    \hbox{\reset@font\missingcitation}%
            \G@refundefinedtrue
        \@latex@warning
        {Citation `\@citeb' on page \thepage \space undefined}}%
        {\@cite@ofmt{\csname b@\@citeb\endcsname}}}}{#1}}
    \makeatother
    \end{verbatim}
\end{minipage}
\begin{minipage}[t]{0.18\textwidth}
    This is a missing citation: \cite{somebib}. This command is not compatible with \texttt{natbib}
\end{minipage}


\section{Geometry settings}
\pagediagram
\printinunitsof{cm}{\pagevalues}

\verb|\marginparwidth|: \printinunitsof{cm}\prntlen{\marginparwidth}

\verbatiminput{Pkgs/Geometry}



\end{document}


    
